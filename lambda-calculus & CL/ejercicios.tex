\documentclass{report}
\usepackage[utf8]{inputenc}
\usepackage[a4paper,top=4cm,bottom=4cm,left=3cm,right=3cm,marginparwidth=1.75cm]{geometry}
\usepackage[spanish]{babel}
\usepackage{libertinus}		
\usepackage{inconsolata}	
\usepackage{amsthm,amsmath,amsfonts,derivative,cancel}
\usepackage{tikz,graphicx,titling,fancyhdr,framed} 
\usepackage{listings}
\usepackage{xcolor}
\usepackage{changepage}
\usepackage{mdframed}


%%%%%%%%%%%%%%%%%%%%%%%%%%%%%%%
% Esto es para los policiclos %
%%%%%%%%%%%%%%%%%%%%%%%%%%%%%%%
\usepackage{standalone}
\usepackage{tikz}
\usetikzlibrary{positioning, arrows.meta, shapes.geometric}
\newcommand{\polycycle}[2][5]{
  \begin{tikzpicture}[
      >=Stealth,
      node style/.style={draw, ellipse, align=center, minimum height=1cm}
    ]
    
    \foreach \label [count=\i] in {#2} {
      \node[node style] (n\i) at ({90+360/#1*(\i-1)}:3cm) {\label};
    }

    \foreach \i in {1,...,#1} {
      \pgfmathtruncatemacro{\next}{mod(\i,#1)+1}
      \draw[->] (n\i) -- (n\next);
    }
  \end{tikzpicture}%
}

\newcommand{\polycycleShort}[2][5]{
  \begin{tikzpicture}[
      >=Stealth,
      node style/.style={draw, ellipse, align=center, minimum height=1cm}
    ]
    
    \foreach \label [count=\i] in {#2} {
      \node[node style] (n\i) at ({90+360/#1*(\i-1)}:1cm) {\label};
    }

    \foreach \i in {1,...,#1} {
      \pgfmathtruncatemacro{\next}{mod(\i,#1)+1}
      \draw[->] (n\i) -- (n\next);
    }
  \end{tikzpicture}%
}
%%%%%%%%%%%%%%%%%%%%%%%%%%%%%%%


\newcounter{exercise}[chapter]
\renewcommand{\thesection}{Ejercicio \arabic{exercise}}
\newcommand{\exercise}{%
  \refstepcounter{exercise}%
  \section{}%
}

\theoremstyle{definition}
\newtheorem*{solution}{Solución}
\newtheorem*{theorem}{Teorema}
\newtheorem*{proposition}{Proposición}


\lstdefinestyle{haskellstyle}{
    language=Haskell,
    mathescape=false,
    escapeinside={}{},
    texcl=false,
    basicstyle=\ttfamily\footnotesize,
    keywordstyle=\bfseries\color{orange!70!black},
    commentstyle=\itshape\color{teal!90!black},
    stringstyle=\color{olive},
    numbers=left,
    numbersep=10pt,
    stepnumber=1,
    numbersep=8pt,
    frame=single,
    framesep=6pt,
    postbreak=\mbox{\textcolor{gray}{$\hookrightarrow$}\space},
    breaklines=true,
    breakatwhitespace=true,
    tabsize=2,
    showstringspaces=false,
    morekeywords={where,let,in,until,case,of,if,then,else,deriving,instance,class,newtype}
}

\title{Lógica y Programación}
\author{Alejandro Egea López}
\date{\today}

\begin{document}

\pagestyle{fancy}
\setlength{\headheight}{13pt}
\lhead[]{Lógica Y Programación}
\rhead[]{Curso 2025/26}
\cfoot{\rule{\linewidth}{0.4pt} \thepage}

\begin{titlepage} 
	\centering
	\scshape 
	\vspace*{\baselineskip} 
	
	\rule{\textwidth}{1.6pt}\vspace*{-\baselineskip}\vspace*{2pt} % Thick horizontal rule
	\rule{\textwidth}{0.4pt}

	\vspace{0.75\baselineskip}
	{\LARGE Lógica Y Programación\\}
	\vspace{0.3\baselineskip} 
	
	\rule{\textwidth}{0.4pt}\vspace*{-\baselineskip}\vspace{3.2pt} % Thin horizontal rule
	\rule{\textwidth}{1.6pt}
	
	\vspace{2\baselineskip}

	{\Large Problemas de $\lambda$-Calculus y Lógica Combinatoria} 
	
	\vspace{1\baselineskip}

	\includegraphics[width=0.5\textwidth]{logo.png}\par
	
	\vspace{2\baselineskip}
	Universidad de Granada\\
	\vspace{0.5\baselineskip}
	{\scshape\Large Lógica Y Programación}
	
	\vspace{2\baselineskip}
	Curso\\
	\vspace{0.5\baselineskip}
	{\scshape\Large 5to curso}

	\vspace{2\baselineskip}
	Docente\\
	\vspace{0.5\baselineskip}
	{\scshape\Large Francisco Miguel García Olmedo}
	
	\vspace{2\baselineskip}
	Autores\\
	\vspace{0.5\baselineskip}
	{\scshape\Large Alejandro Egea López \\ Nicolás Ramírez Rodiles}
	
	\vfill
	A fecha de\\
	\today
\end{titlepage}

\pagebreak

\tableofcontents

%%%%%%%%%%%%%%%%%%%%%%%%%%%%%%%%%%%%%%%%%%%%%%%%%%%%%%%%%%%%%%%%%%%%%%%%%%%%%%%%%%%%%%
\chapter{Problemas de $\lambda$-Calculus y Lógica Combinatoria}

% EJERCICIO 1
\exercise 
\noindent Exponga y desarrolle justificadamente el tema de la ``Notación de de Bruijn''.\\

\noindent\hrulefill\\

%%%%%%%%%%%%%%%%%%%%%%%%%%%%%%%%%%%%%%%%%%%%%%%%%%%%%%%%%%%%%%%%%%%%%%%%%%%%%%
\renewcommand{\thesection}{\thechapter.\arabic{section}}
\subsection{Introducción}

La notación tradicional del cálculo $\lambda$ usa nombres para las variables ligadas
($x,y,z,\dots$). Esta práctica introduce tres dificultades fundamentales:

\begin{enumerate}
    \item \textbf{Captura accidental}: al sustituir una expresión por una variable libre,
          puede producirse captura si surgen colisiones de nombre con variables ligadas
          en el contexto.
    \item \textbf{Conversión-$\alpha$ obligatoria}: para evitar capturas debe
          renombrarse sistemáticamente las variables ligadas.
    \item \textbf{Complejidad en meta-demostraciones}: propiedades como la
          confluencia o la prueba de Church-Rosser se complican debido a la
          constante necesidad de renombrar.
\end{enumerate}

La propuesta de N.\,G. de Bruijn consiste en \emph{eliminar los nombres de variables
ligadas} y reemplazarlos por \textbf{índices naturales} que indican cuántos ligadores
separan la ocurrencia de su $\lambda$ correspondiente. De este modo:

\begin{itemize}
    \item desaparece por completo la conversión-$\alpha$,
    \item no hay colisiones de nombres,
    \item las definiciones de sustitución, reducción $\beta$ y $\eta$ se vuelven
          puramente estructurales,
    \item las demostraciones metateóricas se simplifican sustancialmente.
\end{itemize}

\noindent El propio de Bruijn establece tres criterios para evaluar una notación:
\begin{enumerate}
    \item[(i)] legibilidad humana,
    \item[(ii)] claridad en discusión metalingüística,
    \item[(iii)] utilidad para implementaciones automáticas.
\end{enumerate}
La notación de índices sacrifica parcialmente el punto (i), pero destaca en (ii) y (iii).


%%%%%%%%%%%%%%%%%%%%%%%%%%%%%%%%%%%%%%%%%%%%%%%%%%%%%%%%%%%%%%%%%%%%%%%%%%%%%%
\subsection{Name-carrying expressions y árbol sintáctico}

Antes de eliminar nombres, representamos aplicaciones mediante un símbolo especial:
\[
A((M),(N)),
\]
que corresponde a la aplicación usual $MN$. Esto permite un análisis uniforme de la estructura.

Consideremos ahora la siguiente expresión $\lambda$-cálculo (tomada del ejemplo del artículo):

\[
\boxed{
\lambda x\ a(\lambda b(x,t,f(\lambda u\ a(u,t,z),\ \lambda s\ w)),w,y)
}
\]

En forma estructurada ``name-carrying’’, cada rama del árbol lleva etiquetas $a,b,f$
según la construcción.

El artículo presenta el siguiente \textbf{árbol anotado} con \emph{reference depth} y
\emph{level}. Los pares $(d,\ell)$ representan:

\begin{itemize}
    \item $d =$ \emph{reference depth}: número de ligadores $\lambda$ desde la
          ocurrencia hasta su ligador,
    \item $\ell =$ \emph{level}: número total de $\lambda$ hasta la raíz.
\end{itemize}

\begin{center}
\includegraphics[width=0.35\textwidth]{tree.png}\par  
\end{center}


%%%%%%%%%%%%%%%%%%%%%%%%%%%%%%%%%%%%%%%%%%%%%%%%%%%%%%%%%%%%%%%%%%%%%%%%%%%%%%
\subsubsection*{Lectura precisa del árbol}

Tomemos algunos nodos para ejemplificar:

\begin{itemize}
    \item El nodo $x$ tiene etiqueta $(2,2)$:
        \begin{itemize}
            \item reference depth $=2$: está bajo dos $\lambda$ antes de llegar a su
                  ligador original \(\lambda_x\),
            \item level $=2$: desde la ocurrencia hasta la raíz hay dos $\lambda$.
        \end{itemize}
    \item El nodo $t$ bajo $b(x,t)$ tiene $(1,2)$:
        \begin{itemize}
            \item depth = 1: está inmediatamente bajo $\lambda_t$,
            \item level = 2: hay dos ligadores superiores en total.
        \end{itemize}
    \item El nodo $v$ etiquetado $(3,1)$ indica que:
        \begin{itemize}
            \item depth = 3: su ligador está tres $\lambda$ más arriba,
            \item level = 1: solo un $\lambda$ separa esa rama de la raíz.
        \end{itemize}
    \item Lo mismo ocurre para $u,t,z,w$ en la rama derecha bajo $f$.
\end{itemize}

Este árbol es la base para convertir la expresión a notación sin nombres.

%%%%%%%%%%%%%%%%%%%%%%%%%%%%%%%%%%%%%%%%%%%%%%%%%%%%%%%%%%%%%%%%%%%%%%%%%%%%%%
\subsection{De nombres a índices: notación de de Bruijn}

Para cada ocurrencia ligada, sustituimos su nombre por su \emph{reference depth}.

\begin{itemize}
    \item Las variables libres se conservan en una lista ordenada
          $(x_1,x_2,\dots)$,
    \item Los \emph{level} sirven como información auxiliar para verificar la
          corrección formal, pero se omiten en la expresión final.
\end{itemize}

\subsubsection*{Ejemplos simples}

\[
\lambda x.\,x \;\longmapsto\; \lambda.\,1
\]

\[
\lambda x.\,\lambda y.\,(x\,y) \;\longmapsto\; \lambda.\,\lambda.\,(2\,1)
\]

\[
\lambda x.\,\lambda y.\,\lambda z.\,x \;\longmapsto\; \lambda.\,\lambda.\,\lambda.\,3
\]

%%%%%%%%%%%%%%%%%%%%%%%%%%%%%%%%%%%%%%%%%%%%%%%%%%%%%%%%%%%%%%%%%%%%%%%%%%%%%%
\subsection{Sintaxis de expresiones sin nombres}

Usamos una gramática claramente indentada:

\begin{verbatim}
<NF>         ::= <Const>
               | <Index>
               | <NFList>
               | lambda.<NF>

<Const>      ::= a | b | c | ...         ; constantes simbólicas

<Index>      ::= 1 | 2 | 3 | ...         ; variable ligada por índice

<NFList>     ::= A(<NF>, <NF>)           ; aplicación
               | <NF> <NF>               ; concatenación de expresiones
\end{verbatim}

Esta sintaxis es extremadamente regular: ya no aparecen nombres ligados, sólo índices.


%%%%%%%%%%%%%%%%%%%%%%%%%%%%%%%%%%%%%%%%%%%%%%%%%%%%%%%%%%%%%%%%%%%%%%%%%%%%%%
\subsection{Sustitución en notación de de Bruijn}

La sustitución se define como:

\[
S(Z_1,Z_2,\dots;\; Q),
\]

donde $Z_i$ sustituye a la variable libre cuyo índice es $i$.

Casos fundamentales:

\begin{itemize}
    \item Si $Q$ es una constante, permanece igual.
    \item Si $Q=k$ es un índice, entonces:
          \[
            S(\dots,Z_k;\;k) = Z_k,
          \]
          con ajuste de índices si la sustitución entra bajo una $\lambda$.
    \item Si $Q=A((Q_1),(Q_2))$, entonces:
          \[
             S(Z;\,A((Q_1),(Q_2)))
             = A((S(Z;Q_1)),(S(Z;Q_2))).
          \]
    \item Si $Q=\lambda.\,R$, entonces:
          \[
             S(Z_1,Z_2,\dots;\;\lambda.\,R)
             = \lambda.\,S(Z'_1,Z'_2,\dots;\;R),
          \]
          donde cada $Z'_i$ es la versión de $Z_i$ con sus índices incrementados
          en uno (para preservar referencias correctas).
\end{itemize}

%%%%%%%%%%%%%%%%%%%%%%%%%%%%%%%%%%%%%%%%%%%%%%%%%%%%%%%%%%%%%%%%%%%%%%%%%%%%%%
\subsection{Reducción $\beta$}

La regla de contracción $\beta$ se expresa elegantemente:

\[
A((\lambda.\,Q),(r))
\;\longrightarrow\;
S(r,2,3,\dots;\;Q),
\]

es decir: sustituimos el índice $1$ por $r$.

\subsubsection*{Ejemplos}

\[
(\lambda x.\,x)\;a
\quad\leadsto\quad
\lambda.\,1\; a
\;\longrightarrow\; a.
\]

\[
(\lambda x.\,\lambda y.\,x)\;y
\quad\leadsto\quad
A((\lambda.\,\lambda.\,2),(y))
\;\longrightarrow\;
\lambda.\,1.
\]

En ningún momento aparece conversión-$\alpha$.

%%%%%%%%%%%%%%%%%%%%%%%%%%%%%%%%%%%%%%%%%%%%%%%%%%%%%%%%%%%%%%%%%%%%%%%%%%%%%%
\subsection{Reducción $\eta$}

La regla extensional:

\[
\lambda.\,A((Q),1)\;\longrightarrow\;Q
\]

siempre que $Q$ no contenga ninguna referencia al índice $1$ ligado por la $\lambda$.

Ejemplo:

\[
\lambda x.\,f\,x
\quad\longmapsto\quad
\lambda.\,A((f),1)
\quad\longrightarrow\quad
f.
\]

%%%%%%%%%%%%%%%%%%%%%%%%%%%%%%%%%%%%%%%%%%%%%%%%%%%%%%%%%%%%%%%%%%%%%%%%%%%%%%
\subsection{Algunos comentarios sobre la reducción múltiple}

De Bruijn desarrolla una teoría de \emph{reducción múltiple} $\beta_U$, donde un conjunto
$U$ de símbolos de aplicación se reduce simultáneamente.

Ideas centrales:

\begin{itemize}
    \item se define la noción de \emph{expresión $U$-correcta},  
    \item la sustitución conserva $U$-corrección (Teorema 10.1),  
    \item la reducción múltiple y la sustitución poseen reglas claras de
          conmutación y composición (Teorema 11.1),  
    \item reducciones múltiples para conjuntos distintos $U,V$ conmutan
          entre sí (Teorema 11.2),  
    \item todo ello conduce a una demostración pulcra del
          \textbf{teorema de Church-Rosser} en la notación sin nombres.
\end{itemize}

En la notación de índices no existe la conversión-$\alpha$ ni conflictos de nombres,
por lo que la demostración se vuelve más transparente: la confluencia se logra por
propiedades puramente estructurales.

%%%%%%%%%%%%%%%%%%%%%%%%%%%%%%%%%%%%%%%%%%%%%%%%%%%%%%%%%%%%%%%%%%%%%%%%%%%%%%
\subsection{Conclusión}

La notación de de Bruijn resuelve de raíz los problemas de captura y renombrado al
sustituir nombres ligados por índices estructurales. La sustitución y la reducción
se vuelven definiciones limpias, algebraicas y mecanizables.  

La aplicación a la reducción múltiple y a la prueba de Church-Rosser muestra que la
notación no sólo simplifica cálculos locales, sino que también clarifica la teoría
global del cálculo $\lambda$.


\pagebreak
% EJERCICIO 2
\renewcommand{\thesection}{Ejercicio \arabic{exercise}}
\exercise 
\noindent Demuestre que para todo $\lambda$-término $N$, \; $\lambda x.x\,K\,N \;\# \; \lambda x.x\,S\,N$. (Nota: Recuérdese que $S\equiv\lambda xyz.xz(yz)$ y $K\equiv\lambda xy.x$).
\begin{solution}
Supongamos, por reducción al absurdo, que $\lambda x.x\,K\,N = \lambda x.x\,S\,N$ y vamos a llegar a la contradicción de que $K=S$, lo cual es imposible porque \( S\#K \) por clase.\\

\noindent En primer lugar, aplicamos la ``regla 3'' con \( Z=K \), de modo que 
\[ \lambda x.x\,K\,N \; = \; \lambda x.x\,S\,N \Rightarrow (\lambda x.x\,K\,N)K \; = \; (\lambda x.x\,S\,N)K \]
Ahora bien, aplicando ``$\beta$-conversión'' y ``sustitución 1'' obtenemos que 
\[ (\lambda x.x\,K\,N)K = KKN \qquad \text{ y } \qquad (\lambda x.x\,S\,N)K = KSN \]
tal que, por transitividad, obtenemos \[ KKN = KSN \]
Finalmente, basta con aplicar la propia definición de K, concluyendo por transitividad que \[ KKN=K \qquad \text{ y } \qquad KSN = S \]
\[ \Rightarrow K=S \]
lo cual es una contradicción.
\end{solution}

\pagebreak
% EJERCICIO 3
\exercise 
\noindent Dibuje razonadamente el grafo $G_\beta(WWW)$, donde $W\equiv\lambda xy.xyy$.
\begin{solution}
Primero, es interesante recordar la definición del grafo de $\beta$-conversión de un $\lambda$-término tal que 

\begin{mdframed}[linewidth=1pt, backgroundcolor=gray!10, roundcorner=5pt]
El grafo de $\beta$-conversión de un $\lambda$-término T (o grafo de reducción de un $\lambda$-término T), que denotaremos por \( G_\beta(T) \), verifica:
\begin{enumerate}
  \item Un $\lambda$-término $M$ es un nodo de \( G_\beta(T) \) si \( \lambda \vdash T = M  \). 
  \item Si \( M_1,M_2 \) son nodos distintos de \( G_\beta(T) \), entonces \( M_1 \not\equiv M_2 \) (en particular, admite \( M_1 =_\beta M_2 \)).
  \item \( n\geq 1 \) aristas unen nodo \( M_1 \) a nodo \( M_2 \) (pudiendo ser \( M_1 \equiv M_2 \)) si, y solo si, \( \lambda \vdash M_1 = M_2 \)
\end{enumerate}
\end{mdframed}

\noindent Comencemos por reducir el $\lambda$-término $(WWW)$: 
\begin{align*}
  WWW & \equiv (\lambda xy.xyy)WW \\
  & = (\lambda y.xyy[x:=W])W \qquad\text{por $\beta$-conversión}\\
  & = (\lambda y.Wyy)W \qquad\text{por sustitución (4) } \\
  & = (Wyy [y:=W]) \qquad\text{por $\beta$-conversión}\\
  & = WWW \qquad\text{por sustitución (4) } \\
\end{align*} 
En efecto, tenemos dos $\lambda$-términos no iguales (en el sentido de $\equiv$). Y \( G_\beta(WWW) \) es

\begin{center}
\polycycleShort[2]{
  $\mathrm{WWW}$,
  $(\lambda y.Wyy)W$
}
\end{center}

\end{solution}

\pagebreak
% EJERCICIO 4
\exercise 
\noindent Encuentre razonadamente un $\lambda$-término $M$ tal que $G_\beta(M)$ sea exactamente:

\begin{solution}
Sea el candidato \( M \equiv NNI \) donde \[ N \equiv \lambda xy.yyyxxy \qquad\text{e}\qquad I \equiv \lambda x.x \] y veamos que \( G_\beta(M) \) es el grafo del enunciado. Sabiendo que \( II \equiv (\lambda x.x)I = I \), tenemos que
\begin{align*}
  NNI & \equiv (\lambda xy.yyyxxy)NI \\
  & = (\lambda y.yyyNNy)I \\
  & = IIINNI \\ 
  & = IINNI \\
  & = INNI \\
  & = NNI \qquad\text{(¡hemos cerrado el ciclo!)}  
\end{align*} 
En efecto, tenemos cinco $\lambda$-términos no iguales (en el sentido de $\equiv$). Y \( G_\beta(M) \) es

\begin{center}
\polycycle[5]{
  $\mathrm{NNI}$,
  $(\lambda y.yyyNNy)I$,
  $IIINNI$,
  $IINNI$,
  $INNI$
}
\end{center}

Vamos a explorar un poco más este $\lambda$-término que hemos propuesto. Presentamos las siguientes definiciones: 
\[ N_k = \lambda xy.\underbrace{y \cdots y}_{\text{k veces}}xxy \qquad\text{y}\qquad C_{k+2}=N_k N_k I \qquad\text{para } k \in \mathbb{N} \]

Así pues, estas definiciones nos conducen a plantear el siguiente lema.

\begin{mdframed}[linewidth=1pt, backgroundcolor=gray!10, roundcorner=5pt]
  \begin{center}
    \( G_\beta(C_{k+2}) \) es un grafo de reducción cíclico de \( k+2 \) nodos para todo \( k \in \mathbb{N} \).
  \end{center}
\end{mdframed}

Por una parte, veamos el caso base \( k=0 \) tal que \( N_0=\lambda xy.xxy \) y por ende, \( C_2=N_0 N_0 I \). Tras la primera parte del ejercicio, es inmediato comprobar que \( (\lambda y.N_0 N_0 y)I = N_0 N_0 I \) se trata de \( G_\beta(C_2) \) cíclico de 2 nodos. Cabe destacar que el caso base constituye un caso \textit{degenerado}, puesto que \textit{un ciclo de dos nodos no es más que el segmento que los une}.

\begin{center}
\polycycleShort[2]{
  $\mathrm{N_0N_0I}$,
  $(\lambda y.N_0N_0y)I$
}
\end{center}
\end{solution}

Por otra parte, veamos el caso para un $k>0$ arbitrario tal que \( N_k = \lambda xy.\underbrace{y \cdots y}_{\text{k veces}}xxy \), es decir
\begin{align*}
  C_{k+2} = N_k N_k I = (\lambda xy.\underbrace{y \cdots y}_{\text{k veces}}xxy) N_k I & = \\
  & =  (\lambda y.\underbrace{y \cdots y}_{\text{k veces}}N_kN_ky) I \\
  & = \underbrace{I \cdots I}_{\text{k veces}}N_kN_k I \\
  & = \underbrace{I \cdots I}_{\text{k-1 veces}}N_kN_k I \\
  & = \dots \\
  & = I N_kN_k I \\
  & = N_kN_kI
\end{align*}

En conclusión, es inmediato observar que hay \( k+2 \) $\lambda$-términos no iguales (en el sentido de $\equiv$) y así podemos hacer que \( G_\beta(C_{k+2}) \) sea un policiclo de cualquier número \textbf{finito} de nodos mayor o igual que 2.

\begin{center}
\polycycle[5]{
  $\mathrm{N_kN_kI}$,
  $(\lambda y.\underbrace{y \cdots y}_{\text{k veces}}N_kN_ky) I$,
  $\underbrace{I \cdots I}_{\text{k veces}}N_kN_k I$,
  $\dots$,
  $I N_kN_k I$
}
\end{center}

\pagebreak
% EJERCICIO 5
\exercise 
\noindent Sea el $\lambda$-término: \[ G \equiv \lambda yx.x(yx) \qquad\text{y}\qquad M \equiv (\lambda xy.y(xxy))(\lambda xy.y(xxy)).\]
\begin{enumerate}
\item Demuestre que $M$ es un punto fijo de $G$.
\item Demuestre que si el combinador $N$ es un punto fijo de $G$, entonces $N$ es un operador de punto fijo.
\item Demuestre que $M$ es un combinador de punto fijo.
\item Demuestre que si $M$ es un combinador de punto fijo, entonces $M = GM$.
\end{enumerate}
\begin{solution}
\mbox{} \\

\textbf{(1)} Si aplicamos el Teorema del Punto Fijo al $\lambda$-término G, sean $y$ tal que \( y \notin FV(G) \), \( W=\lambda z.G(zz) \) y \( X=WW \), entonces obtenemos que \( X = M \) es punto fijo de $G$. En efecto, 
\begin{align*}
  X & \equiv \\
  & \equiv WW \\
  & = (\lambda yx.x(yyx))(\lambda yx.x(yyx)) \qquad\text{por $(\dagger)$}\\
  & \equiv (\lambda xy.y(xxy))(\lambda xy.y(xxy)) \qquad\text{por $\alpha$-congruencia}
\end{align*}
donde \((\dagger)\) resulta por
\begin{align*}
  W & \equiv \\
  & \equiv \lambda y.G(yy) \\
  & \equiv \lambda y.(\lambda yx.x(yx))(yy) \\
  & = \lambda y.((\lambda x.x(yx))[y:=yy]) \qquad\text{por $\beta$-conversión}\\
  & \equiv \lambda y.\lambda x.x(yyx) \qquad\text{por sustitución (1)}\\
  & = \lambda yx.x(yyx)
\end{align*}

\textbf{(2)} Supongamos que el combinador $N$ es un punto fijo de $G$, es decir, \( N = GN \). Por esto,
\begin{align*}
  N & = \\
  & = GN \\
  &\equiv (\lambda yx.x(yx))N \\
  &= \lambda x.((x(yx))[y:=N]) \qquad\text{por $\beta$-conversión}\\
  &\equiv \lambda x.x(Nx) \qquad\text{por sustitución 1}\\
\end{align*}
Y concluimos que, dado $F$ un $\lambda$-término,
\begin{align*}
  NF & = \\
  & = (\lambda x.x(Nx))F \\
  & = (x(Nx))[x:=F] \qquad\text{por $\beta$-conversión}\\
  & = F(NF) \qquad\text{por sustitución 1}
\end{align*}
que es precisamente lo que se quería demostrar. \\

\textbf{(3)} Vamos a presentar una forma inmediata de resolver el ejercicio y otra más didáctica.

Por una parte, $M$ es claramente un combinador porque ninguna variable ocurra libre y, además, es punto fijo de $G$ por el apartado (1). Tenemos que verifica las hipótesis del apartado (2) y por tanto, $M$ es un operador (o combinador) de punto fijo. 

Por otra parte, si denotamos por \( W:=\lambda xy.y(xxy) \) tenemos que 
\begin{align*}
  M & = \\
  & \equiv WW \\
  & \equiv (\lambda xy.y(xxy))W \\
  & = \lambda y.((y(xxy))[x:=W]) \qquad\text{por $\beta$-conversión}\\
  & \equiv \lambda y.y(WWy) \qquad\text{por sustitución 1}\\
  & \equiv \lambda y.y(My) \qquad\text{denotamos por \((\dagger)\)}
\end{align*}
Y llegamos a que, dado $F$ un $\lambda$-término,
\begin{align*}
  MF & = \\
  & = (\lambda y.y(My))F \qquad\text{por \((\ddagger)\)}\\
  & = (y(My))[y:=F] \qquad\text{por $\beta$-conversión}\\
  & \equiv F(MF)  \qquad\text{por sustitución 1}\\
\end{align*}
es decir, que $M$ es un combinador de punto fijo.\\

\textbf{(4)} Supongamos que $N$ es un combinador de punto fijo, es decir, para todo $F$ $\lambda$-término \( NF=F(NF) \). Tomando el $G$ del enunciado tenemos que 
\begin{align*}
  GM & = \\
  & \equiv (\lambda yx.x(yx))N \\
  & = \lambda x.((x(yx))[y:=N]) \qquad\text{por $\beta$-conversión}\\
  & \equiv \lambda x.x(Nx)  \qquad\text{por sustitución 1}\\
\end{align*}

Ahora bien, si tomamos \( T=GN \) llegamos a que
\begin{align*}
  TF = \\
  & = GNF \qquad\text{por regla 3}\\
  & = (\lambda x.x(Nx))F \\
  & = (x(Nx))[x:=F] \qquad\text{por $\beta$-conversión}\\
  & = F(NF) \qquad\text{por sustitución 1}\\
\end{align*}
Finalmente, aplicamos la hipótesis obteniendo que \( TF=F(NF)=NF \), y de nuevo, por la regla 3, tenemos que \( T=N \).

Como partíamos de que \( T=GN \), efectivamente llegamos a que \( N=GN \), es decir, $N$ es punto fijo de $G$.

\end{solution}


\pagebreak
% EJERCICIO 6
\exercise 
\noindent Considere el combinador:
\[ Y\equiv \lambda y.(\lambda x.y(xx))(\lambda x.y(xx)) \]
y demuestre que $GY = Y$.
\begin{solution}
Probemos que $Y$ es un combinador de punto fijo para que, aplicando el Ejercicio 5 (d), concluyamos que \( Y=GY \).
Es decir, debemos probar que para cada $F$ $\lambda$-término se tiene que \( YF=F(YF) \), que es inmediato por $\beta$-conversión tal que 
\begin{align*}
  YF & \equiv \\
  & \equiv (\lambda y.(\lambda x.y(xx))(\lambda x.y(xx)))F \\
  & = ((\lambda x.y(xx))(\lambda x.y(xx)))[y:=F] \qquad\text{por $\beta$-conversión}\\
  & \equiv (\lambda x.y(xx))[y:=F](\lambda x.y(xx))[y:=F] \qquad\text{por sustitución 6}\\
  & \equiv (\lambda x.F(xx))(\lambda x.F(xx)) \qquad\text{por sustitución 1}\\
  & \equiv \omega \omega 
\end{align*}
Y ahora bastaría continuar reduciendo \( YF=\omega \omega \) tal que
\begin{align*}
  \omega \omega & = \\
  & = F(xx)[x:=\omega] \qquad\text{por $\beta$-conversión}\\
  & \equiv F(\omega\omega) \qquad\text{por sustitución 1}\\
  & \equiv F(YF) \qquad\text{por el cálculo previo}\\
\end{align*}
Por lo tanto, hemos probado que $Y$ es combinador de punto fijo, con lo cual afirmamos que \( Y=GY \).
\end{solution}

\pagebreak
% EJERCICIO 7
\exercise 
\noindent Considere la sucesión de combinadores $\{Y^n\}_n$ definida para todo número natural $n$ como sigue:
\[ Y^n = \begin{cases} Y, &\text{si } n=0, \\ Y^{n-1}G, &\text{si } n>0. \end{cases} \]
Demuestre que para todo $n\ge 0$, $Y^n$ es un combinador de punto fijo.
\begin{solution}
Razonamos por inducción sobre \( n\geq 0 \). Por una parte, el caso base \( \boxed{n=0} \) es \( Y^0=Y \), que es claramente un combinador de punto fijo por el Ejercicio 6.

Por otra parte, veamos el caso inductivo \( \boxed{S(k) \Rightarrow S(k+1) \, \forall k>0} \) donde la hipótesis de inducción es \( S(k)=\text{«$Y^k$es un combinador de punto fijo»} \). Queremos demostrar que \( S(k+1) \), y procedemos de la siguiente manera 
\[ Y^{k+1} \overset{def.}{=} Y^k G \overset{H.I.}{=} G(Y^k G) \overset{def.}{=} G(Y^{k+1})  \]
En efecto, como \( Y^{k+1} = G(Y^{k+1}) \), entonces \( Y^{k+1} \) es un combinador de punto fijo como se quería demostar, completando la inducción.
\end{solution}

\pagebreak
% EJERCICIO 8
\exercise 
\noindent Encuentre razonadamente el CL-término $(\lambda xy.xyy)_{CL}$.
\begin{solution}
En primer lugar, escribimos el $\lambda$-término como un CL-término de tipo `[x].M' (que usaremos solo permitir la conversión $\lambda \to CL$, pero que no es un CL-término en sí mismo):
\[ (\lambda xy.xyy)_{CL} \to [xy].xyy, \]
y, por definición de abstracción de múltiples variables, obtenemos que 
\[ [xy].xyy \equiv [x]([y].xyy) \]
de modo que ya podemos proceder a aplicar las reglas de abstracción (a), (b), (c), (f) (las que se requieran). A modo de recordatorio:

\begin{mdframed}[linewidth=1pt, backgroundcolor=gray!10, roundcorner=5pt]
\textbf{Definition 2.18 (Abstraction).}
Para todo término de CL llamémoslo $M$ y toda variable $x$, se define por inducción un término de CL llamado $[x].M$ del siguiente modo:
\begin{enumerate}
  \item[(a)] $[x].M \equiv K M$ si $x \notin FV(M)$.
  \item[(b)] $[x].x \equiv I$.
  \item[(c)] $[x].Ux \equiv U$ si $x \notin FV(U)$.
  \item[(f)] $[x].UV \equiv S([x].U)([x].V)$ si no se aplica ninguna de las reglas anteriores.
\end{enumerate}
*Cabe destacar que la referencia prescinde de las reglas (d) y (e); y que en [x].UV usando la regla (f), V es siempre la parte derecha de la aplicación más externa del término.
\end{mdframed}
Vamos a operar según las reglas (*las reglas se enuncian en orden de aplicación):
\begin{align*}
  [x].([y].xyy) & \equiv \\
  & \equiv [x].(S([y].xy)([y].y))  \qquad\text{por regla (f)}\\
  & \equiv [x].(S(S([y].x)([y].y))I)  \qquad\text{por regla (f) y (b)}\\
  & \equiv [x].(S(S(Kx)(I))I)  \qquad\text{por regla (a) y (b)}\\
  & \equiv S([x].S(S(Kx)(I)))([x].I)  \qquad\text{por regla (f)}\\
  & \equiv S(S([x].S)([x].(S(Kx)(I))))KI  \qquad\text{por regla (f) y (a)}\\
  & \equiv S(S(KS)([x].(S(Kx)(I))))KI  \qquad\text{por regla (a) y (f)}\\
  & \equiv S(S(KS)(S([x].S(Kx))([x].I)))KI  \qquad\text{por regla (f)}\\
  & \equiv S(S(KS)(S(S([x].S)([x].Kx))KI))KI  \qquad\text{por regla (f) y (a)}\\
  & \equiv S(S(KS)(S(S(KS)K)KI))KI  \qquad\text{por regla (a) y (c)}\\
\end{align*}

Y por último, podemos aplicar las propias definiciones de los combinadores S, K, I para simplificar la expresión tal que
\begin{align*}
  S(S(KS)(S(S(KS)K)KI))KI & \equiv \\
  & \equiv \text{FALTA TERMINAR ESTA SIMPLIFACIÓN}
\end{align*}


\end{solution}

\pagebreak
% EJERCICIO 9
\exercise 
\noindent Esquematice la relación entre el sistema $\lambda$ y la lógica combinatoria.

\noindent\hrulefill\\

\renewcommand{\thesection}{\thechapter.\arabic{section}}
\subsection{Introducción}

El sistema $\lambda$ y la lógica combinatoria son dos sistemas equivalentes en poder expresivo, pero con distintos enfoques. El $\lambda$-cálculo usa variables, abstracción $(\lambda x.M)$, sustitución y $\alpha$-conversión mientras que la lógica combinatoria trabaja únicamente con aplicación y combinadores ($S, K, I$, etc.). Para profundizar más en cómo se relacionan, se hará una definición formal de la lógica combinatoria para después relacionarlo con el sistema $\lambda$.

\subsection{Términos y combinadores de CL}

Los sistemas de combinadores realizan la misma funcionalidad que el $\lambda$-cálculo sin el uso de variables ligadas. Un combinador es un término cerrado de la lógica combinatoria, es decir, un término que no contiene variables y que se construye únicamente mediante aplicación y constantes combinatorias. Hay un número infinito de combinadores de los cuales tres se denominan combinadores básicos:
\begin{itemize}
    \item \textbf{I}, el operador identidad: $\textbf{I}(f) = f$
    \item \textbf{K}, combinador que forma funciones constantes: $(\textbf{K}(a))(x) = a$
    \item \textbf{S}, el operador de composición: $(\textbf{S}(f,g))(x) = f(x,g(x))$
\end{itemize}

\noindent El alfabeto de los términos de la lógica combinatoria es el compuesto por los siguientes símbolos: 
\begin{itemize}
    \item símbolos de variable $v_0, v_{00}, v_{000}$
    \item constantes atómicas, incluyendo los combinadores básicos.  
\end{itemize}
El conjunto de términos de la lógica combinatoria se define inductivamente como sigue:

\begin{enumerate}
  \renewcommand{\labelenumi}{\alph{enumi})}
  \item Toda variable y toda constante atómica es un término;
  \item si $X$ e $Y$ son términos,  entonces $(XY)$ también lo es.
\end{enumerate}

Cualquier sucesión finita de elementos del alfabeto de términos es un término de la lógica combinatoria. Por ejemplo:
\[ \textbf{((S(KS))K)},\qquad\qquad\textbf{((S($Kv_0$))((SK)K))}\]son ambos términos de la lógica combinatoria, siendo el primero además un combinador. 

\noindent Conociendo los términos podemos definir la relación \textit{X ocurre en Y}, o \textit{X es un subtérmino de Y}: 
\begin{enumerate}
  \renewcommand{\labelenumi}{\alph{enumi})}
  \item $X$ ocurre en $X$
  \item si $X$ ocurre en $U$ o en $V$, entonces $X$ ocurre en $(UV)$
\end{enumerate}
Es importante destacar que todas las variables que ocurran en un término $Y$ pertenecen a FV($Y$) puesto que no hay ningún $\lambda$ que las capture.

\subsection{Sustitución y sustitución débil}

Al igual que en el $\lambda$-cálculo, la lógica combinatoria también define la sustitución de términos de forma recursiva. $[U/X]Y$ se define recursivamente como el resultado de sustituir el término $U$ por cada ocurrencia de $x$ en $Y$ de la siguiente forma: 

\begin{enumerate}
  \renewcommand{\labelenumi}{\alph{enumi})} 
  \item $[U/x]x \equiv U,$
  \item $[U/x]a \equiv a$ si $a \not\equiv x$
  \item $[U/x](VW) \equiv ([U/x]V[U/x]W). $ 
\end{enumerate}

La lógica combinatoria carece de $\beta$ reducción, por lo que la computación se define mediante reglas de reducción para \textbf{I, K} y \textbf{S}. Cualquier término $\textbf{I}X$, $\textbf{K}YZ$ o $\textbf{S}XYZ$ se denomina un \textit{redex débil. Contraer} un redex débil significa reducir los términos de la siguiente forma:  
\begin{itemize}
    \item $\mathbf{I}X \to X$,
    \item $\mathbf{K}XY \to X$,
    \item $\mathbf{S}XYZ \to XZ(YZ)$.
\end{itemize}
Si mediante una de estas reglas $U$ pasa a $U'$, se dice que \textit{U se contrae débilmente a U'} expresado notacionalmente como 
\[U\to_{1w}U'\]
Si $U'$ se obtiene a partir de una sucesión finita de reducciones débiles, se dice que \textit{reduce débilmente a U'}, o  
\[U\to_wU'\]

\subsection{Eliminación de la abstración $\lambda$}

Dado que la lógica combinatoria carece de abstracción $\lambda$, es necesario introducir un mecanismo que permita representar funciones con parámetros. Esto se logra mediante la construcción meta-teórica $[x].M$ que traduce la abstracción del cálculo lambda a un término de lógica combinatoria con la propiedad de que \[([x].M)N\to_w[N/x]M.\]
El término teórico $[x].M$, que no es un CL-término en sí mismo, actuará de forma equivalente a $\lambda x.M$ y se creará como una combinación de los términos \textbf{I, K} y \textbf{S} de la siguiente forma: 
\begin{enumerate}
  \renewcommand{\labelenumi}{\alph{enumi})}
  \item $[x].M \equiv \mathbf{K}M \qquad \text{si } x \notin FV(M)$;
  \item $[x].x \equiv \mathbf{I}$;
  \item $[x].(Ux) \equiv U \qquad \text{si } x \notin FV(U)$;
  \item $[x].(UV) \equiv \mathbf{S}([x].U)([x].V) \qquad \text{si no se aplican (a) ni (c).}$
\end{enumerate}

\noindent Por ejemplo: 
 \begin{align*}
  [x].(xy) &\equiv \textbf{S}(([x].x)([x].y) \qquad\text{por regla (f)}\\
  & \equiv \textbf{SI}\text{(}\textbf{K}\text{y)} \qquad \text{por (b) y (a).}\\
\end{align*}
\noindent
Así, la propiedad fundamental de la abstracción, esto es,
\[
([x].M)N \to_w [N/x]M,
\]
muestra que el término combinatorio $[x].M$ reproduce exactamente el comportamiento de la abstracción $\lambda x.M$ bajo aplicación, siendo la reducción débil el análogo de la $\beta$-reducción del cálculo lambda.

\subsection{Conclusión}

En consecuencia, toda abstracción del cálculo $\lambda$ puede eliminarse sistemáticamente mediante la construcción $[x].M$, obteniendo un término de la lógica combinatoria formado únicamente por aplicación y combinadores básicos. Este procedimiento permite traducir cualquier término del cálculo lambda a un término combinatorio que preserva su comportamiento computacional.

Por tanto, la lógica combinatoria constituye una reformulación del cálculo $\lambda$ sin variables ligadas ni abstracción explícita, pero con el mismo poder expresivo. Ambos sistemas describen la misma clase de funciones computables, diferenciándose únicamente en el enfoque sintáctico utilizado para representar la noción de función.

%%%%%%%%%%%%%%%%%%%%%%%%%%%%%%%%%%%%%%%%%%%%%%%%%%%%%%%%%%%%%%%%%%%%%%%%%%%%%%%%%%%%%%
\chapter{Unidad Temática: Lógica de Primer Orden — Los Teoremas de incompletitud de Gödel}

\renewcommand{\thesection}{\thechapter.\arabic{section}}

\section{Introducción}
Esta unidad temática presenta una exposición formal pero accesible de los Teoremas de Incompletitud de Gödel. El objetivo es combinar rigor matemático —mediante notación y resultados precisos— con explicaciones conceptuales que faciliten la comprensión de las ideas fundamentales.

\section{Consistencia, coherencia y completitud}

\subsection{Consistencia}
Sea $\Phi$ un conjunto de fórmulas de un lenguaje lógico $L$. Una \emph{contradicción} es una fórmula de la forma $\varphi \land \neg\varphi$.

\begin{proposition}[\emph{Ex falso quodlibet}]
Para un conjunto de fórmulas $\Phi$ son equivalentes:
\begin{enumerate}
\item Existe una fórmula $\varphi$ tal que $\Phi \vdash \varphi \land \neg\varphi$.
\item Para toda fórmula $\psi$, se tiene $\Phi \vdash \psi$.
\end{enumerate}
\end{proposition}

Este principio expresa que, en presencia de una contradicción, el sistema pierde todo poder discriminatorio. Por ello, diremos que $\Phi$ es \emph{inconsistente} si demuestra alguna contradicción, y \emph{consistente} en caso contrario.

\subsection{Coherencia}
La relación entre demostrabilidad sintáctica y verdad semántica se recoge en el siguiente resultado:

\begin{theorem}[Teorema de la Coherencia]
Sea $\Phi$ un conjunto de fórmulas y $\mathcal{M}$ un modelo tal que $\mathcal{M} \models \Phi$. Entonces, para toda fórmula $\varphi$:
\[
\Phi \vdash \varphi ; \Rightarrow ; \mathcal{M} \models \varphi.
\]
\end{theorem}

Este teorema garantiza que el sistema lógico es \emph{correcto}: nada falso en un modelo puede ser demostrado a partir de las hipótesis verdaderas en dicho modelo.

\subsection{Completitud}
Una teoría $T$ se dice \emph{completa} si para toda fórmula cerrada $\varphi$ se cumple:
\[
T \vdash \varphi \quad \text{o bien} \quad T \vdash \neg\varphi.
\]

El siguiente resultado clásico conecta consistencia, modelos y demostrabilidad:

\begin{theorem}[Teorema de Completitud]
Sea $\Phi$ un conjunto consistente de fórmulas. Entonces existe un modelo $\mathcal{M}$ tal que $\mathcal{M} \models \Phi$. Además, para toda sentencia $\sigma$:
\[
(\forall \mathcal{M} \models \Phi,; \mathcal{M} \models \sigma) \Rightarrow \Phi \vdash \sigma.
\]
\end{theorem}

Este resultado puede interpretarse como que toda verdad semántica universal es demostrable, siempre que el sistema sea consistente.

\section{Axiomática de Peano}

\subsection{Axiomas de Peano}

\begin{align*}
\text{AP0: }& \neg \exists x,(s(x) = 0) \\
\text{AP1: }& \forall x\forall y,(s(x) = s(y) \rightarrow x = y) \\
\text{AP2: }& \forall x,(x + 0 = x) \\
\text{AP3: }& \forall x\forall y,(x + s(y) = s(x + y)) \\
\text{AP4: }& \forall x,(x \cdot 0 = 0) \\
\text{AP5: }& \forall x\forall y,(x \cdot s(y) = (x \cdot y) + x)
\end{align*}


\subsection{Aritmética y propiedades de AP}

\section{La Función de Gödel}


\section{Teoremas de Incompletitud de Gödel}

\subsection{Lema de la diagonalización}

\begin{theorem}[Lema de la diagonalización]
Sea $\varphi(\nu)$ una fórmula con una única variable libre. Existe una sentencia $\sigma$ tal que:
\[
\text{AP} \vdash \sigma \leftrightarrow \varphi(\ulcorner \sigma \urcorner).
\]
\end{theorem}

Este lema permite construir sentencias autorreferenciales, pieza central de la demostración de los teoremas de Gödel.

\subsection{Primer Teorema de Incompletitud}

\begin{theorem}[Primer Teorema de Incompletitud]
Si la Axiomática de Peano es consistente, entonces es incompleta.
\end{theorem}

La idea de la demostración consiste en construir una sentencia que afirma su propia no demostrabilidad. Si dicha sentencia fuese demostrable o refutable, se obtendría una contradicción con la consistencia del sistema.

\subsection{Segundo Teorema de Incompletitud}

\begin{theorem}[Segundo Teorema de Incompletitud]
Si la Axiomática de Peano es consistente, entonces no puede demostrar su propia consistencia.
\end{theorem}

Este resultado muestra una limitación aún más profunda: ningún sistema formal suficientemente expresivo puede, desde dentro, garantizar formalmente que está libre de contradicciones.

\section{Conclusión}
Los Teoremas de Incompletitud de Gödel establecen límites fundamentales a los sistemas formales y, por extensión, a los métodos algorítmicos. En informática, estos resultados subyacen a la imposibilidad de construir procedimientos automáticos universales para demostrar todas las verdades matemáticas o verificar completamente sistemas complejos.



%%%%%%%%%%%%%%%%%%%%%%%%%%%%%%%%%%%%%%%%%%%%%%%%%%%%%%%%%%%%%%%%%%%%%%%%%%%%%%%%%%%%%%
\chapter{Ejercicios de Programación en Haskell}

\renewcommand{\thesection}{Ejercicio \arabic{exercise}}
\setcounter{exercise}{1}

\exercise

\begin{lstlisting}[style=haskellstyle, caption=Cálculo del área de un triángulo según la Fórmula de Herón]
import Data.List (sort)

heron :: (Ord a, Fractional a) => a -> a -> a -> Maybe a
heron a b c
  | not $ isTriangle a b c = Nothing
  | otherwise              = Just . newton f df $ x0
  where
    s = (a+b+c)/2
    t = s*(s-a)*(s-b)*(s-c)
    f x = x*x-t
    df x = 2*x
    x0 = max 1 $ t/2

    isTriangle :: (Ord a, Num a) => a -> a -> a -> Bool 
    isTriangle a b c = x>0 && x+y>z
      where [x,y,z] = sort [a,b,c]

    newton :: (Ord a, Fractional a) => (a -> a) -> (a -> a) -> a -> a
    newton f df x0 = until (isCloseEnough 1e-12) nextIteration x0
      where
        isCloseEnough eps x = abs (f x) < eps
        nextIteration x = x - f x / df x
\end{lstlisting}

\pagebreak

\exercise 

\begin{lstlisting}[style=haskellstyle, caption=Criptosistema de Vigenère]
import Data.Char (isLetter, toUpper, ord, chr)
import Data.List (group, sort, nub, elemIndices, sortOn, maximumBy)
import Control.Applicative (ZipList(..))

class Monoid g => Group g where
    invert :: g -> g

------------------------- Datos y Algebras -------------------------

data Mode = Encrypt | Decrypt

newtype Shift = Shift Int deriving (Eq, Show)

instance Semigroup Shift where
    (Shift a) <> (Shift b) = Shift ((a + b) `mod` 26)

instance Monoid Shift where
    mempty = Shift 0

instance Group Shift where
    invert (Shift n) = Shift (-n `mod` 26)

--------------------- Logica de transformacion ---------------------

applyMode :: Mode -> Shift -> Shift
applyMode Encrypt = id
applyMode Decrypt = invert

normalize :: String -> String
normalize str = toUpper <$> filter isLetter str

shiftChar :: Char -> Shift -> Char
shiftChar c (Shift n) = chr (ord 'A' + (ord c - ord 'A' + n) `mod` 26)

----------------------- Algoritmo de Vigenere ----------------------

vigenereTransform :: Mode -> String -> String -> String
vigenereTransform mode key str = getZipList $ 
    shiftChar <$> ZipList (str) 
              <*> ZipList (cycle shifts)
  where
    shifts = map toShift $ key
    toShift c = applyMode mode $ Shift (ord c - ord 'A')

encrypt :: String -> String -> String
encrypt key str = vigenereTransform Encrypt cKey cStr
  where
    cKey = normalize key 
    cStr = normalize str

decrypt :: String -> String -> String
decrypt = vigenereTransform Decrypt

----------------------- Metodo de Kasiski --------------------------

factorize :: Int -> [Int]
factorize n = sort . concat $ [ [x, n `div` x] | x <- [1..limit], n `mod` x == 0 ]
  where 
    limit = floor . sqrt . fromIntegral $ n

possibleLengths :: String -> [Int]
possibleLengths str = concatMap (take 1) . sortOn (negate . length) . group $ sortedFactors
  where
    trigrams = zip3 str (drop 1 str) (drop 2 str)
    dists = do
        (x:_) <- filter ((>1) . length) $ group $ sort trigrams
        let indices = elemIndices x trigrams
        zipWith (-) (drop 1 indices) indices
    sortedFactors = sort . filter (\k -> k > 1 && k <= 20) . concatMap factorize $ dists

--------------------- Indice de Coincidencia -----------------------

buildSubstrings :: Int -> String -> [String]
buildSubstrings k msg = [ [ msg !! j | j <- [i, i+k .. length msg - 1] ] | i <- [0 .. k-1]]

frequencies :: String -> [Int]
frequencies s = map length . group . sort $ s

indexOfCoincidence :: String -> Double
indexOfCoincidence s
  | n <= 1    = 0
  | otherwise = fromIntegral acc / fromIntegral (n * (n - 1))
  where
    freqs = frequencies s
    acc   = sum [ f * (f - 1) | f <- freqs ]
    n     = sum freqs
    
indexOfCoincidenceTest :: Int -> String -> [Double]
indexOfCoincidenceTest k message = map indexOfCoincidence substrings
  where
    substrings = buildSubstrings k (message)

textoCifrado :: String
textoCifrado =
  "UECWKDVLOTTVACKTPVGEZQMDAMRNPDDUXLBUICAMRHOECBHSPQLVIWO\
  \FFEAILPNTESMLDRUURIFAEQTTPXADWIAWLACCRPBHSRZIVQWOFROGTT\
  \NNXEVIVIBPDTTGAHVIACLAYKGJIEQHGECMESNNOCTHSGGNVWTQHKBPR\
  \HMVUOYWLIAFIRIGDBOEBQLIGWARQHNLOISQKEPEIDVXXNETPAXNZGDX\
  \WWEYQCTIGONNGJVHSQGEATHSYGSDVVOAQCXLHSPQMDMETRTMDUXTEQQ\
  \JMFAEEAAIMEZREGIMUECICBXRVQRSMENNWTXTNSRNBPZHMRVRDYNECG\
  \SPMEAVTENXKEQKCTTHSPCMQQHSQGTXMFPBGLWQZRBOEIZHQHGRTOBSG\
  \TATTZRNFOSMLEDWESIWDRNAPBFOFHEGIXLFVOGUZLNUSRCRAZGZRTTA\
  \YFEHKHMCQNTZLENPUCKBAYCICUBNRPCXIWEYCSIMFPRUTPLXSYCBGCC\
  \UYCQJMWIEKGTUBRHVATTLEKVACBXQHGPDZEANNTJZTDRNSDTFEVPDXK\
  \TMVNAIQMUQNOHKKOAQMTBKOFSUTUXPRTMXBXNPCLRCEAEOIAWGGVVUS\
  \GIOEWLIQFOZKSPVMEBLOHLXDVCYSMGOPJEFCXMRUIGDXNCCRPMLCEWT\
  \PZMOQQSAWLPHPTDAWEYJOGQSOAVERCTNQQEAVTUGKLJAXMRTGTIEAFW\
  \PTZYIPKESMEAFCGJILSBPLDABNFVRJUXNGQSWIUIGWAAMLDRNNPDXGN\
  \PTTGLUHUOBMXSPQNDKBDBTEECLECGRDPTYBVRDATQHKQJMKEFROCLXN\
  \FKNSCWANNAHXTRGKCJTTRRUEMQZEAEIPAWEYPAJBBLHUEHMVUNFRPVM\
  \EDWEKMHRREOGZBDBROGCGANIUYIBNZQVXTGORUUCUTNBOEIZHEFWNBI\
  \GOZGTGWXNRHERBHPHGSIWXNPQMJVBCNEIDVVOAGLPONAPWYPXKEFKOC\
  \MQTRTIDZBNQKCPLTTNOBXMGLNRRDNNNQKDPLTLNSUTAXMNPTXMGEZKA\
  \EIKAGQ"

------------------- Mutual Index Of Coincidence --------------------

spanishFreq :: [(Char, Double)]
spanishFreq =
  [ ('A',0.1253),('B',0.0142),('C',0.0468),('D',0.0586)
  , ('E',0.1368),('F',0.0069),('G',0.0101),('H',0.0070)
  , ('I',0.0625),('J',0.0044),('K',0.0002),('L',0.0497)
  , ('M',0.0315),('N',0.0671),('O',0.0868),('P',0.0251)
  , ('Q',0.0088),('R',0.0687),('S',0.0798),('T',0.0463)
  , ('U',0.0393),('V',0.0090),('W',0.0001),('X',0.0022)
  , ('Y',0.0090),('Z',0.0052)
  ]

relativeFreq :: Char -> String -> Double
relativeFreq c s = fromIntegral (length (filter (== c) s)) / fromIntegral (length s)

mutualIndex :: String -> Double
mutualIndex s = sum [ p * relativeFreq c s | (c,p) <- spanishFreq ]

micForShift :: String -> Shift -> Double
micForShift s b = mutualIndex (map (`shiftChar`b) s)

bestShift :: String -> Shift
bestShift s = snd $ maximumBy cmp results
  where
    results = [ (micForShift s b, b) | b <- map Shift [0..25] ]
    cmp (x,_) (y,_) = compare x y

shiftToKeyChar :: Shift -> Char
shiftToKeyChar b = shiftChar 'A' (invert b)

mutualIndexTest :: Int -> String -> String
mutualIndexTest k msg = map (shiftToKeyChar . bestShift) substrings
  where
    substrings = buildSubstrings k (msg)

----------------------------- Attack -------------------------------

attackVigenere :: String -> (String, String)
attackVigenere cipherText = (clave, textoDescifrado)
  where
    normText = normalize cipherText
    candidates = possibleLengths normText
    bestK = maximumBy compareIndexOfCoincidence candidates
    clave = mutualIndexTest bestK normText
    textoDescifrado = decrypt clave normText
    compareIndexOfCoincidence k1 k2 = compare (avgIC k1) (avgIC k2)
      where
        avgIC k = let result = indexOfCoincidenceTest k normText 
                  in sum result / fromIntegral (length result)
\end{lstlisting}

\pagebreak

\begin{thebibliography}{9}

\bibitem{deBruijn1972}
De Bruijn, N. G. (1972).
\textit{Lambda calculus notation with nameless dummies, a tool for automatic formula manipulation, with application to the Church–Rosser theorem}.
\textit{Indagationes Mathematicae (Proceedings), 75}(5), 381--392.
https://doi.org/10.1016/1385-7258(72)90034-0

\bibitem{venturini1984-reduction}
Venturini Zilli, M. (1984).
\textit{Reduction graphs in the lambda calculus}.
\textit{Theoretical Computer Science, 29}(3), 251--275.
https://doi.org/10.1016/0304-3975(84)90002-1

\bibitem{hindley2008-cl}
Hindley, J. R., \& Seldin, J. P. (2008).
\textit{Lambda-calculus and combinators: An introduction} (2nd ed.).
Cambridge University Press.
(Secciones relevantes: Section 2C, \textit{Abstraction in CL}, pp.~26--29; Chapter 9, \textit{Correspondence between $\lambda$ and CL}, pp.~92--106.)

\bibitem{garcia2023-godel}
García López, J. (2023).
\textit{Los teoremas de incompletitud de Gödel (Gödel’s Incompleteness Theorems)}.
Trabajo de Fin de Grado, Grado en Matemáticas,
Universidad de Cantabria.
{https://hdl.handle.net/10902/29859}.



\end{thebibliography}


\end{document}
