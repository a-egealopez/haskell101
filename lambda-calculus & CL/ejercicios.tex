\documentclass{article}
\usepackage[utf8]{inputenc}
\usepackage[a4paper,top=4cm,bottom=4cm,left=3cm,right=3cm,marginparwidth=1.75cm]{geometry}
\usepackage[spanish]{babel}	
\usepackage{libertinus}		
\usepackage{inconsolata}	
\usepackage{amsthm,amsmath,amsfonts,derivative,cancel}
\usepackage{tikz,graphicx,titling,fancyhdr,framed} 
\usepackage{listings}
\usepackage{xcolor}
\usepackage{changepage}
\usepackage{standalone}

\newtheorem{exercise}{Ejercicio}
\newtheorem{solution}{Solución}

\lstdefinestyle{plaintextstyle}{ 
  basicstyle=\footnotesize,        % Fija el tamaño del tipo de letra utilizado para el código
  breakatwhitespace=false,         % Activarlo para que los saltos automáticos solo se apliquen en los espacios en blanco
  breaklines=true,                 % Activa el salto de línea automático
  captionpos=b,                    % Establece la posición de la leyenda del cuadro de código
  extendedchars=true,              % Permite utilizar caracteres extendidos no-ASCII; solo funciona para codificaciones de 8-bits; para UTF-8 no funciona. En xelatex necesita estar a true para que funcione.
  frame=single,	                   % Añade un marco al código
  keepspaces=true,                 % Mantiene los espacios en el texto. Es útil para mantener la indentación del código(puede necesitar columns=flexible).
  numbers=left,                    % Posición de los números de línea (none, left, right).
  numbersep=10pt,                   % Distancia de los números de línea al código
  numberstyle=\small\color{black}, % Estilo para los números de línea
  rulecolor=\color{black},         % Si no se activa, el color del marco puede cambiar en los saltos de línea entre textos que sea de otro color, por ejemplo, los comentarios, que están en verde en este ejemplo
  showspaces=false,                % Si se activa, muestra los espacios con guiones bajos; sustituye a 'showstringspaces'
  showstringspaces=false,          % subraya solamente los espacios que estén en una cadena de esto
  showtabs=false,                  % muestra las tabulaciones que existan en cadenas de texto con guión bajo
  stepnumber=1,                    % Muestra solamente los números de línea que corresponden a cada salto. En este caso: 1,3,5,...
  tabsize=2,	                   % Establece el salto de las tabulaciones a 2 espacios
  title=\lstname,                  % muestra el nombre de los ficheros incluidos al utilizar \lstinputlisting; también se puede utilizar en el parámetro caption
}

\lstdefinestyle{cppstyle}{ 
  language=C++, 
  basicstyle=\ttfamily\small,       % Fuente monoespaciada pequeña
  keywordstyle=\bfseries\color{teal},        % Palabras clave en azul
  commentstyle=\bfseries\color{purple},       % Comentarios en verde
  stringstyle=\bfseries\color{olive},          % Cadenas en rojo
  numbers=left,                     % Números de línea a la izquierda
  numberstyle=\tiny\color{black},     % Estilo de los números de línea en rojo
  stepnumber=1,                     % Numerar cada línea
  numbersep=5pt,                    % Separación entre el número y el código
  frame=single,                     % Marco alrededor del código
  backgroundcolor=\color{white},    % Color de fondo blanco
  showstringspaces=false,           % No mostrar espacios en las cadenas
  tabsize=2,                        % Tamaño de tabulación
  breaklines=true,                  % Habilitar la ruptura de líneas largas
  breakatwhitespace=true,           % Romper líneas solo en espacios
  rulecolor=\color{black},         % Si no se activa, el color del marco puede cambiar en los saltos de línea entre textos que sea de otro color, por ejemplo, los comentarios, que están en verde en este ejemplo
  emph={verificar,contenida,equiparar,resolver,log,to_tab,
        combinarEvidencias,juntar_evidencias_hasta,combinarReglaEvidencia,
        juntar_reglas_hasta,anadir_meta,to_decimal},  % Funciones específicas a resaltar
  emphstyle=\bfseries\color{violet}           % Estilo para funciones en violeta
}

\title{Lógica y Programación}
\author{Alejandro Egea López}
\date{\today}

\begin{document}

\pagestyle{fancy}
\setlength{\headheight}{13pt}
\lhead[]{Lógica Y Programación}
\rhead[]{Curso 2025/26}
\cfoot{\rule{\linewidth}{0.4pt} \thepage}

\begin{titlepage} 
	\centering
	\scshape 
	\vspace*{\baselineskip} 
	
	\rule{\textwidth}{1.6pt}\vspace*{-\baselineskip}\vspace*{2pt} % Thick horizontal rule
	\rule{\textwidth}{0.4pt}

	\vspace{0.75\baselineskip}
	{\LARGE Lógica Y Programación\\}
	\vspace{0.3\baselineskip} 
	
	\rule{\textwidth}{0.4pt}\vspace*{-\baselineskip}\vspace{3.2pt} % Thin horizontal rule
	\rule{\textwidth}{1.6pt}
	
	\vspace{2\baselineskip}

	{\Large Problemas de $\lambda$-Calculus y Lógica Combinatoria} 
	
	\vspace{1\baselineskip}

	\includegraphics[width=0.5\textwidth]{logo.png}\par
	
	\vspace{2\baselineskip}
	Universidad de Granada\\
	\vspace{0.5\baselineskip}
	{\scshape\Large Lógica Y Programación}
	
	\vspace{2\baselineskip}
	Curso\\
	\vspace{0.5\baselineskip}
	{\scshape\Large 5to curso}

	\vspace{2\baselineskip}
	Docente\\
	\vspace{0.5\baselineskip}
	{\scshape\Large Francisco Miguel García Olmedo}
	
	\vspace{2\baselineskip}
	Autores\\
	\vspace{0.5\baselineskip}
	{\scshape\Large Alejandro Egea López \\ Nicolás Ramírez Rodiles}
	
	\vfill
	A fecha de\\
	\today
\end{titlepage}

\pagebreak

\tableofcontents

\pagebreak
% EJERCICIO 1
\begin{exercise}[Notación de de Bruijn]
1. Exponga y desarrolle justificadamente el tema de la ``Notación de de Bruijn''.\\
\textbf{Pistas:} Defina índices de de Bruijn, explique la conversión de variables ligadas a índices (y viceversa), muestre reglas para renombrado (\textit{alpha-conversion}) y ejemplos resueltos. Discuta ventajas (eliminación del conflicto de nombres) y limitaciones.
\begin{solution}

\end{solution}
\end{exercise}

\pagebreak
% EJERCICIO 2
\begin{exercise}[Relación entre términos y combinadores K,S]
Demuestre que para todo $\lambda$-término $N$, \; $\lambda x.x\,K\,N \;\vdash\; \lambda x.x\,S\,N$.\\
(Nota: recuerde que $S\equiv\lambda xyz.xz(yz)$ y $K\equiv\lambda xy.x$.)
\begin{solution}

\end{solution}
\end{exercise}

\pagebreak
% EJERCICIO 3
\begin{exercise}[Grafo de un término]
Dibuje razonadamente el grafo $G(W W W)$, donde $W\equiv\lambda xy.xyy$.
\begin{solution}

\end{solution}
\end{exercise}

\pagebreak
% EJERCICIO 4
\begin{exercise}[Construcción de término por grafo]
Encuentre razonadamente un $\lambda$-término $M$ tal que $G(M)$ sea exactamente:\\
\begin{solution}

\end{solution}
\end{exercise}

\pagebreak
% EJERCICIO 5
\begin{exercise}[Punto fijo y combinadores]
Sea el $\lambda$-término: \[ G \equiv \lambda yx:x(yx) \qquad\text{y}\qquad M \equiv (\lambda xy:y(xxy))(\lambda xy:y(xxy)).\]
Demostrar:
\begin{enumerate}
\item Demuestre que $M$ es un punto fijo de $G$.
\item Demuestre que si el combinador $N$ es un punto fijo de $G$, entonces $N$ es un operador de punto fijo.
\item Demuestre que $M$ es un combinador de punto fijo.
\item Demuestre que si $M$ es un combinador de punto fijo, entonces $M = GM$.
\end{enumerate}
\begin{solution}

\end{solution}
\end{exercise}

\pagebreak
% EJERCICIO 6
\begin{exercise}[Combinador Y]
Considere el combinador:
\[ Y\equiv \lambda y:(\lambda x:y(xx))(\lambda x:y(xx)) \]
y demuestre que $GY = Y$.
\begin{solution}

\end{solution}
\end{exercise}

\pagebreak
% EJERCICIO 7
\begin{exercise}[Sucesión de combinadores]
Considere la sucesión de combinadores $\{Y_n\}_n$ definida para todo número natural $n$ como sigue:
\[ Y_n = \begin{cases} Y, &\text{si } n=0, \\ Y_{n-1}G, &\text{si } n>0. \end{cases} \]
Demuestre que para todo $n\ge 0$, $Y_n$ es un combinador (o la propiedad que desee demostrar — complete la afirmación según el enunciado original).
\begin{solution}

\end{solution}
\end{exercise}

\pagebreak
% EJERCICIO 8
\begin{exercise}[CL-término]
Encuentre razonadamente el CL-término $(\lambda xy:xyy)_{CL}$.
\begin{solution}

\end{solution}
\end{exercise}

\pagebreak
% EJERCICIO 9
\begin{exercise}[Relación entre sistemas]
Esquematice la relación entre el sistema $\lambda$ y la lógica combinatoria.
\begin{solution}

\end{solution}
\end{exercise}


\end{document}
